\chapter{Introduction}%
\label{chp:introduction}

The ever-evolving landscape of cyber threats poses a pervasive danger to all
facets of society, ranging from commercial enterprises to individuals and
government entities. As technology relentlessly advances and more aspects of
our lives become digitised, the risks posed by these threats continue to
escalate~\cite{ENISA}. There are numerous defence options available to help
mitigate these threats, prominent among them are \gls{nids}.

\gls{nids} are a crucial part of any information system's defence that aims to
identify and mitigate network threats in real time. Two major
classifications exist that characterise the function of intrusion detection
systems.\ \gls{sids} detect attacks by matching sequences of code or commands
with those of known attacks.\ \gls{aids} on the other hand, detect threats by
recognising deviations from normal, non-malicious traffic~\cite{survey1}.

Unknown attacks pose a challenge to these \gls{nids}. These could take the form
of new deviations of known attacks, new attacks based on known vulnerabilities
or zero-day attacks, which are cyberattacks executed on the basis of an unknown
exploit. Due to their nature, \gls{sids} are ineffective in detecting unknown
attacks, as by definition, they cannot exist on any attack signature
database~\cite{survey1}.

\gls{aids} exhibits promise through its ability to identify unknown
attacks~\cite{aids-unseen}, particularly with the advent of \gls{ml}
techniques, notably \gls{dl}. Extensive research exists in this field, however,
many researchers present their work as a closed-set classification
problem~\cite{zero-day}. A closed-set classification problem is one where the
dataset includes all possible classes the model needs to predict. This may
create a false sense of security as this does not account for the possibility
of unknown attacks. Furthermore, these closed-set models are presented as
\gls{aids}, creating the implication that they can detect unknown attacks, with
no empirical evidence to measure the models' efficacy in this regard.
