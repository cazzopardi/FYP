\chapter{Introduction}%
\label{chp:introduction}

The ever-evolving landscape of cyber threats poses a pervasive danger to all
facets of society, ranging from commercial enterprises to individuals and
government entities. As technology relentlessly advances and more aspects of
our lives become digitised, the risks posed by these threats continue to
escalate. There are numerous defence options available to help mitigate these
threats, prominent among them are \gls{ids}.

\gls{ids} are a crucial part of any information system's defence that aims to
identify and mitigate threats in real time. There are two primary domains at
which \gls{ids} can operate, namely \gls{nids}, which analyse
network packets to identify attacks, and \gls{hids}, which analyse logs and
other information sources from a single host. In addition, two major
classifications exist that characterise the function of intrusion detection
systems.\ \gls{sids} detect attacks by matching sequences of code or commands
with those of known attacks.\ \gls{aids} on the other hand, detect threats by
recognising deviations from normal, non-malicious traffic~\cite{survey1}.

Extensive research exists in the field of \gls{ids}, however, despite these
advancements, zero-day attacks remain an especially challenging problem in the
field of cybersecurity~\cite{zero-day}. Zero-day attacks are cyber attacks
executed on the basis of an exploit that was not previously known. Due to their
nature, \gls{sids} are ineffective in detecting zero-day attacks, as by
definition, they cannot exist on any attack signature database~\cite{survey1}.

\gls{aids} exhibits promise through its ability to identify zero-day
attacks~\cite{aids-zero-day}, particularly with the advent of \gls{ml}
techniques, notably \gls{dl}. Extensive research exists in this field, however,
many researchers present their work as a closed set classification
problem~\cite{zero-day}. A closed-set classification problem is one where the
dataset includes all possible classes the model needs to predict. This may
create a false sense of security as this does not account for the possibility
of zero-day attacks, which have been increasing in prevalence~\cite{symantec2017}.
Futhermore, these closed-set models are presented as \gls{aids}, creating the
implication that they can detect zero-day attacks, with no empirical evidence
to measure the models' efficacy in this regard.
