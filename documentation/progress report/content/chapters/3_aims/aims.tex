\chapter{Aims and Objectives}%
\label{chp:aims}
\hypertarget{obj}{}

The principal aim of this study, will be to empirically investigate the
interaction between current state-of-the-art network intrusion detection models
and unknown attacks. This will take the form of two objectives:

\begin{center}
    \fbox{\parbox{0.8\linewidth}{\textbf{Objective 1:} To determine the extent to which current state-of-the-art
            network intrusion detection models are able to detect attacks that were not
            present in their training set.}}
\end{center}

This information is vital to measure the pragmatic performance of these models
as \gls{nids} deployed in practical circumstances will face a wide variety of
attacks, most of which will not have been present in its training set.

\begin{center}
    \fbox{\parbox{0.8\linewidth}{\textbf{Objective 2:} To investigate the extent to which \gls{dl} is
            more or less effective in detecting unknown attacks than traditional \gls{ml}
            techniques.}}
\end{center}

\gls{dl} is able to model complex patterns and relationships,
requiring far less feature engineering and preprocessing from the developer.
Therefore, we hypothesise these features may make it better equipped to model
the complex and abstract underlying patterns in malicious behaviour. This study
will aim to verify this hypothesis.
